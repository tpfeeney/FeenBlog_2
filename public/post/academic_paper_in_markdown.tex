% Options for packages loaded elsewhere
\PassOptionsToPackage{unicode}{hyperref}
\PassOptionsToPackage{hyphens}{url}
%
\documentclass[
]{article}
\usepackage{lmodern}
\usepackage{amssymb,amsmath}
\usepackage{ifxetex,ifluatex}
\ifnum 0\ifxetex 1\fi\ifluatex 1\fi=0 % if pdftex
  \usepackage[T1]{fontenc}
  \usepackage[utf8]{inputenc}
  \usepackage{textcomp} % provide euro and other symbols
\else % if luatex or xetex
  \usepackage{unicode-math}
  \defaultfontfeatures{Scale=MatchLowercase}
  \defaultfontfeatures[\rmfamily]{Ligatures=TeX,Scale=1}
\fi
% Use upquote if available, for straight quotes in verbatim environments
\IfFileExists{upquote.sty}{\usepackage{upquote}}{}
\IfFileExists{microtype.sty}{% use microtype if available
  \usepackage[]{microtype}
  \UseMicrotypeSet[protrusion]{basicmath} % disable protrusion for tt fonts
}{}
\makeatletter
\@ifundefined{KOMAClassName}{% if non-KOMA class
  \IfFileExists{parskip.sty}{%
    \usepackage{parskip}
  }{% else
    \setlength{\parindent}{0pt}
    \setlength{\parskip}{6pt plus 2pt minus 1pt}}
}{% if KOMA class
  \KOMAoptions{parskip=half}}
\makeatother
\usepackage{xcolor}
\IfFileExists{xurl.sty}{\usepackage{xurl}}{} % add URL line breaks if available
\IfFileExists{bookmark.sty}{\usepackage{bookmark}}{\usepackage{hyperref}}
\hypersetup{
  pdftitle={Writing an Academic Paper in RMarkdown},
  pdfauthor={TF},
  hidelinks,
  pdfcreator={LaTeX via pandoc}}
\urlstyle{same} % disable monospaced font for URLs
\usepackage[margin=1in]{geometry}
\usepackage{color}
\usepackage{fancyvrb}
\newcommand{\VerbBar}{|}
\newcommand{\VERB}{\Verb[commandchars=\\\{\}]}
\DefineVerbatimEnvironment{Highlighting}{Verbatim}{commandchars=\\\{\}}
% Add ',fontsize=\small' for more characters per line
\usepackage{framed}
\definecolor{shadecolor}{RGB}{248,248,248}
\newenvironment{Shaded}{\begin{snugshade}}{\end{snugshade}}
\newcommand{\AlertTok}[1]{\textcolor[rgb]{0.94,0.16,0.16}{#1}}
\newcommand{\AnnotationTok}[1]{\textcolor[rgb]{0.56,0.35,0.01}{\textbf{\textit{#1}}}}
\newcommand{\AttributeTok}[1]{\textcolor[rgb]{0.77,0.63,0.00}{#1}}
\newcommand{\BaseNTok}[1]{\textcolor[rgb]{0.00,0.00,0.81}{#1}}
\newcommand{\BuiltInTok}[1]{#1}
\newcommand{\CharTok}[1]{\textcolor[rgb]{0.31,0.60,0.02}{#1}}
\newcommand{\CommentTok}[1]{\textcolor[rgb]{0.56,0.35,0.01}{\textit{#1}}}
\newcommand{\CommentVarTok}[1]{\textcolor[rgb]{0.56,0.35,0.01}{\textbf{\textit{#1}}}}
\newcommand{\ConstantTok}[1]{\textcolor[rgb]{0.00,0.00,0.00}{#1}}
\newcommand{\ControlFlowTok}[1]{\textcolor[rgb]{0.13,0.29,0.53}{\textbf{#1}}}
\newcommand{\DataTypeTok}[1]{\textcolor[rgb]{0.13,0.29,0.53}{#1}}
\newcommand{\DecValTok}[1]{\textcolor[rgb]{0.00,0.00,0.81}{#1}}
\newcommand{\DocumentationTok}[1]{\textcolor[rgb]{0.56,0.35,0.01}{\textbf{\textit{#1}}}}
\newcommand{\ErrorTok}[1]{\textcolor[rgb]{0.64,0.00,0.00}{\textbf{#1}}}
\newcommand{\ExtensionTok}[1]{#1}
\newcommand{\FloatTok}[1]{\textcolor[rgb]{0.00,0.00,0.81}{#1}}
\newcommand{\FunctionTok}[1]{\textcolor[rgb]{0.00,0.00,0.00}{#1}}
\newcommand{\ImportTok}[1]{#1}
\newcommand{\InformationTok}[1]{\textcolor[rgb]{0.56,0.35,0.01}{\textbf{\textit{#1}}}}
\newcommand{\KeywordTok}[1]{\textcolor[rgb]{0.13,0.29,0.53}{\textbf{#1}}}
\newcommand{\NormalTok}[1]{#1}
\newcommand{\OperatorTok}[1]{\textcolor[rgb]{0.81,0.36,0.00}{\textbf{#1}}}
\newcommand{\OtherTok}[1]{\textcolor[rgb]{0.56,0.35,0.01}{#1}}
\newcommand{\PreprocessorTok}[1]{\textcolor[rgb]{0.56,0.35,0.01}{\textit{#1}}}
\newcommand{\RegionMarkerTok}[1]{#1}
\newcommand{\SpecialCharTok}[1]{\textcolor[rgb]{0.00,0.00,0.00}{#1}}
\newcommand{\SpecialStringTok}[1]{\textcolor[rgb]{0.31,0.60,0.02}{#1}}
\newcommand{\StringTok}[1]{\textcolor[rgb]{0.31,0.60,0.02}{#1}}
\newcommand{\VariableTok}[1]{\textcolor[rgb]{0.00,0.00,0.00}{#1}}
\newcommand{\VerbatimStringTok}[1]{\textcolor[rgb]{0.31,0.60,0.02}{#1}}
\newcommand{\WarningTok}[1]{\textcolor[rgb]{0.56,0.35,0.01}{\textbf{\textit{#1}}}}
\usepackage{graphicx}
\makeatletter
\def\maxwidth{\ifdim\Gin@nat@width>\linewidth\linewidth\else\Gin@nat@width\fi}
\def\maxheight{\ifdim\Gin@nat@height>\textheight\textheight\else\Gin@nat@height\fi}
\makeatother
% Scale images if necessary, so that they will not overflow the page
% margins by default, and it is still possible to overwrite the defaults
% using explicit options in \includegraphics[width, height, ...]{}
\setkeys{Gin}{width=\maxwidth,height=\maxheight,keepaspectratio}
% Set default figure placement to htbp
\makeatletter
\def\fps@figure{htbp}
\makeatother
\setlength{\emergencystretch}{3em} % prevent overfull lines
\providecommand{\tightlist}{%
  \setlength{\itemsep}{0pt}\setlength{\parskip}{0pt}}
\setcounter{secnumdepth}{-\maxdimen} % remove section numbering

\title{Writing an Academic Paper in RMarkdown}
\author{TF}
\date{2020-04-13}

\begin{document}
\maketitle

\hypertarget{background}{%
\subsection{Background:}\label{background}}

I was interested in writing an academic paper in RMarkdown, but I am a
n00b and it wasnt immediately obvious to me how to format everything. So
I did a little searching around and have found two solutions, which I
think are pretty good. One solution, which I'll call the basic YAML
version, requires only some moderate reformatting of YAML header in an
RMarkdown file. The other solution, which I call augemented YAML,
requires some extra files called \texttt{.lua} files, which add extra
features but end up making it look nicer in the end. My goal here was to
end up with a document with a reasonable title page, a single spaced
abstract section with Keywords at the end, and a two colum paper that I
could insert figures into. Also, I needed it to have line numbers.

For this you will need to have a \(LaTeX\) distribution installed,
pandoc, and be able to at least knit to PDF. I use RStudio for all of
this. Knowing some HTML doesnt hurt either.

\hypertarget{basic-yaml}{%
\subsection{Basic YAML}\label{basic-yaml}}

\begin{Shaded}
\begin{Highlighting}[]
\KeywordTok{library}\NormalTok{(knitr)}
\NormalTok{knitr}\OperatorTok{::}\KeywordTok{include\_graphics}\NormalTok{(}\StringTok{"/Users/timf/Documents/Github/academic\_abstract\_in\_markdown/baseline\_yaml/baseline\_yaml.pdf"}\NormalTok{)}
\end{Highlighting}
\end{Shaded}

\begin{figure}
\includegraphics[width=0.85\linewidth]{/Users/timf/Documents/Github/academic_abstract_in_markdown/baseline_yaml/baseline_yaml} \caption{caption}\label{fig:label}
\end{figure}

This is called basic YAML because other than a \texttt{.csl} file for
your references formatting, and a \texttt{.bib} file for your
references, you wont need anything special other than Markdown know how
to write a paper.

\begin{verbatim}
---
title: |-
  This is a wonderful Title
author: 
- Author 1^1,2^
- Author 2^1,2^
- Etc.^1,3^
date: \scriptsize^1^ University of Somewhere ^2^ Imaginary College ^3^ Mans Greatest Hospital
output:
  pdf_document:
  word_document: default
  html_document:
    df_print: paged
#classoption:
#- twocolumn #allows for double column
bibliography:
- all_the_best_refs.bib
header-includes: 
- \usepackage{setspace}\doublespacing # makes for double spacing
- \usepackage[switch, pagewise, running]{lineno} #switch allows it to be used with double column
- \linenumbers # adds line numbers
- \renewcommand\linenumberfont{\normalfont\small} #changes size of line numbers
keywords: 'Keywords: first, Second, Third'
csl: journal-of-the-american-college-of-surgeons.csl #citation format, in this case makes in text numbers, and numbers the ref section.
abstract: |-
 \singlespacing This is my dream abstract. 


 **Keywords: first, second, third**

---
\end{verbatim}

Then after the YAML use:

\begin{verbatim}
\twocolumn
\doublespacing
<!--\newpage-->
\pagewiselinenumbers
\end{verbatim}

To start double spacing, double column, and line numbers

And then at the end near references, to return to single spacing and
single column

\begin{verbatim}
\newpage
\singlespacing
\onecolumn
# References
\end{verbatim}

\hypertarget{augemented-yaml}{%
\subsection{Augemented YAML}\label{augemented-yaml}}

This is `augmented' because additional formatting \texttt{.lua} files
are need, which if you can just keep in the same directory as the
\texttt{.Rmd} file you are writing. For this you will need:

\begin{itemize}
\tightlist
\item
  \texttt{author-info-blocks.lua}
\item
  \texttt{scholarly-metadata.lua}
\end{itemize}

and to knit to word I used:

\begin{itemize}
\tightlist
\item
  \texttt{Reference\_Document.docx}
\end{itemize}

These files can all be obtained at in my
\href{https://github.com/tpfeeney/academic_paper_in_markdown/tree/master/augmented_yaml}{Git
Repo}

\begin{verbatim}
---
title: |
  **A most wonderful title.**
subtitle: |
    _and a nice subtitle_
author:
- First Author:
    correspondence: yes
    email: email@mail.com
    institute:
    - CoM
    - IU
- Second Author:
    correspondence: no
    institute:
    - IU
date: ''
output:
  pdf_document:
    pandoc_args:
    - --filter=pandoc-crossref
    - --lua-filter=scholarly-metadata.lua
    - --lua-filter=author-info-blocks.lua
  word_document: default
  bookdown::word_document2:
    pandoc_args:
    - --filter=pandoc-crossref
    - --lua-filter=scholarly-metadata.lua
    - --lua-filter=author-info-blocks.lua
    - --reference-doc=Reference_Document.docx
bibliography:
- all_the_best_refs.bib
header-includes:
- \usepackage{setspace}\doublespacing
- \usepackage[switch, pagewise, running]{lineno}
- \linenumbers
- \renewcommand\linenumberfont{\normalfont\small}
- \usepackage{rotating}
- \usepackage{float}
institute:
- IU: Imaginary University
- CoM: College of Magic
csl: journal-of-the-american-college-of-surgeons.csl #include your reference format of    choice here.
abstract: |-
 \singlespacing This is my dream abstract. 
  
  
  **Keywords: first, second, third**
Keywords: Incidental Adrenal Mass, Endocrinology,
---
\end{verbatim}

And include the beginning:

\begin{verbatim}
\twocolumn
\doublespacing
<!--\newpage-->
\pagewiselinenumbers
\begin{center}
\textbf{Introduction:}
\end{center}
\end{verbatim}

And ending as before.

\begin{verbatim}
\newpage
\singlespacing
\onecolumn
# References
\end{verbatim}

To include R code and figures/tables after references include this bit

\begin{verbatim}
<div id="refs"></div>

\newpage
\end{verbatim}

\hypertarget{summary}{%
\subsection{Summary}\label{summary}}

I hope this is somewhat helpful. Please see my
\href{https://github.com/tpfeeney/academic_paper_in_markdown}{Git repo}
for output from these and the files needed to make this work.

\hypertarget{future-directions}{%
\subsection{Future Directions}\label{future-directions}}

\begin{itemize}
\tightlist
\item
  Figure out how to include inline tables
\item
  Figure out how to include Tikz graphics and DAGs; likely will need
  \texttt{TikzDevice} for this.
\end{itemize}

\end{document}
